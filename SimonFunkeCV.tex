% Don't like 10pt? Try 11pt or 12pt
\documentclass[11pt]{article}

% This is a helpful package that puts math inside length specifications
\usepackage{calc}
\usepackage{color}
\usepackage[usenames,dvipsnames]{xcolor}
\usepackage[T1]{fontenc}

% Simpler bibsection for CV sections
% (thanks to natbib for inspiration)
\makeatletter
\newlength{\bibhang}
\setlength{\bibhang}{1em}
\newlength{\bibsep}
 {\@listi \global\bibsep\itemsep \global\advance\bibsep by\parsep}
\newenvironment{bibsection}
    {\minipage[t]{\linewidth}\list{}{%
        \setlength{\leftmargin}{\bibhang}%
        \setlength{\itemindent}{-\leftmargin}%
        \setlength{\itemsep}{\bibsep}%
        \setlength{\parsep}{\z@}%
        }}
    {\endlist\endminipage}
\makeatother


\usepackage[paper=letterpaper,
            %includefoot, % Uncomment to put page number above margin
            marginparwidth=0.025in,     % Length of section titles
            marginparsep=.03in,       % Space between titles and text
            margin=0.6in, %45in,               % 1 inch margins
            includemp]{geometry}

%% More layout: Get rid of indenting throughout entire document
\setlength{\parindent}{0in}

%% This gives us fun enumeration environments. compactitem will be nice.
\usepackage{paralist}


\usepackage{fancyhdr,lastpage}
\pagestyle{fancy}
\pagestyle{empty}      % Uncomment this to get rid of page numbers
\fancyhf{}\renewcommand{\headrulewidth}{0pt}
\fancyfootoffset{\marginparsep+\marginparwidth}
\newlength{\footpageshift}
\setlength{\footpageshift}
          {0.5\textwidth+0.5\marginparsep+0.5\marginparwidth-2in}
\lfoot{\hspace{\footpageshift}%
       \parbox{4in}{\, \hfill %
                    \arabic{page} of \protect\pageref*{LastPage} % +LP
%                    \arabic{page}                               % -LP
                    \hfill \,}}

% Finally, give us PDF bookmarks
\usepackage{color,hyperref}
\definecolor{black}{rgb}{0,0,0}
\hypersetup{colorlinks,breaklinks,
            linkcolor= black,urlcolor= black,
            anchorcolor= black,citecolor= black}

%%%%%%%%%%%%%%%%%%%%%%%% End Document Setup %%%%%%%%%%%%%%%%%%%%%%%%%%%%


%%%%%%%%%%%%%%%%%%%%%%%%%%% Helper Commands %%%%%%%%%%%%%%%%%%%%%%%%%%%%
% Place at top of document. It should be the first thing.
\newcommand{\makeheading}[1]%
        {\hspace*{-\marginparsep minus \marginparwidth}%
         \begin{minipage}[t]{\textwidth+\marginparwidth+\marginparsep}%
                {\begin{center} \LARGE \bfseries\textsc{ #1} \end{center}}%[-0.15\baselineskip]%
                 %\rule{\columnwidth}{1pt}%
        \end{minipage}}

\newcommand{\address}[4]%
        {\hspace*{-\marginparsep minus \marginparwidth}%
         \begin{minipage}[t]{\textwidth+\marginparwidth+\marginparsep}%
         {\begin{center} \vspace{0.25cm} \textnormal \bfseries #1 \\ \textit{email: } #2 ~~\textit{web: } #4 ~~\textit{phone: } #3 \end{center}}%[-0.15\baselineskip]%
                 %\rule{\columnwidth}{1pt}%
        \end{minipage}}


\newenvironment{heading}[1]{\hspace{0.5cm} \textbf{#1}}

\newenvironment{headingtitle}[2]{\hspace{0.5cm} \textbf{#1}, #2}

\newenvironment{dateheadingtitle}[2]{ #1 #2.}

\newenvironment{locationdate}[2]{#2 \quad \textbf{#1}.}

\newenvironment{positionlocationdate}[3]{#3 \quad \textbf{#1, #2}.}

\newenvironment{supervision}[3]{#3 \quad \textbf{#1} #2.}

\definecolor{dark-gray}{gray}{0.1}

\newenvironment{sectionheader}[1]{\vspace{0.5cm}\color{black}{\large{\bfseries{\textsc{#1}}}\normalsize
}\color{Black}}

% An itemize-style list with lots of space between items
\newenvironment{outerlist}[1][\enskip\textbullet]
        {\begin{itemize}[ #1]}{\end{itemize}}

% An itemize-style list with little space between items
\newenvironment{innerlist}[1][\enskip\textbullet]
        {\begin{compactitem}[#1]}{\end{compactitem}}

% To add some paragraph space between lines.
% This also tells LaTeX to preferably break a page on one of these gaps
% if there is a needed pagebreak nearby.
\newcommand{\blankline}{\quad\pagebreak[2]}

%%%%%%%%%%%%%%%%%%%%%%%% End Helper Commands %%%%%%%%%%%%%%%%%%%%%%%%%%%

%%%%%%%%%%%%%%%%%%%%%%%%% Begin CV Document %%%%%%%%%%%%%%%%%%%%%%%%%%%%

\begin{document}
\makeheading{Simon Wolfgang Funke}
\address{Simula Research Laboratory, Martin Linges vei 17, 1364 Fornebu, Norway}{simon@simula.no}{+47 40 62 55 86}{\href{http://www.simonfunke.com}{www.simonfunke.com}}

\newlength{\rcollength}\setlength{\rcollength}{1.0in}%

%%%%%%%%%%%%%%%%%%%%%%%%%% Personal information %%%%%%%%%%%%%%%%%%%%%%%%%%%%%%%%
\sectionheader{Personal information}

\begin{itemize}
\item[] Name: Funke, Simon Wolfgang
    \vspace{-0.2cm}
\item[] Date of birth: 17.10.1983
    \vspace{-0.2cm}
\item[] Sex: Male
    \vspace{-0.2cm}
\item[] Nationality: German
    \vspace{-0.2cm}
\end{itemize}

%%%%%%%%%%%%%%%%%%%%%%%%%% Employment %%%%%%%%%%%%%%%%%%%%%%%%%%%%%%%%
\sectionheader{Positions}
\begin{outerlist}
\item[] \positionlocationdate{\hspace{-0.2cm}Research Scientist}{Simula Research Laboratory, Oslo, Norway}{2016 - today}
\item[] \positionlocationdate{\hspace{-0.2cm}MC Member}{COST Action TD1307 (European Model Reduction Network)}{2016 - today}
\item[] \positionlocationdate{\hspace{-0.2cm}Adjoint Associate Professor}{University of Oslo, Norway}{2015 - today}
\item[] \positionlocationdate{\hspace{-0.2cm}Technical advisor (CTO in 2014)}{IMERSO AS, Oslo, Norway}{2014 - today}
\item[] \positionlocationdate{\hspace{-0.2cm}Postdoctoral Fellow}{Simula Research Laboratory, Oslo, Norway}{2013 - 2015}\\
    In addition, a 20\% adjoint PostDoc position in 2014 at Imperial College London, UK.
    %Studying predictive modeling in physiological flow.
\item[] \locationdate{Consultant, Kalkulo AS, Oslo, Norway}{2014 - 2015}
        %Metocean model for oil\&gas industry company.
\item[] \locationdate{Consultant, E.ON AG, D\"usseldorf, Germany}{2013 - 2014}
        %Numerical study of safety measures for the release of toxic dusts.
%\item[] \locationdate{MeyGen }{2013 - 2014}
        %Helped to improve the tidal turbine layouts
\end{outerlist}

%%%%%%%%%%%%%%%%%%%%%%%%%% Employment %%%%%%%%%%%%%%%%%%%%%%%%%%%%%%%%
% Funding:
%Call: Open Disruptive Innovation Scheme (implemented through the SME instrument)
%Position: Lead Developer and CTO
%Duration: 6 months (from December 2015 - May 2016)
%Amount: € 50.000

%%%%%%%%%%%%%%%%%%%%%%%%%% EDUCATION %%%%%%%%%%%%%%%%%%%%%%%%%%%%%%%%
\sectionheader{Education}

\begin{outerlist}
\item[] \positionlocationdate{PhD, Computational science}{Imperial College London, UK}{2009 - 2013}
        \begin{innerlist}
            \item Thesis title: \textit{The automation of PDE-constrained optimisation and its applications}.
            \item Department of Earth Science and Engineering.
            \item Supervisors: M.D. Piggott, P.E. Farrell, P.A. Allison, G.J. Gorman.
            \item Date of approved disputation: 01.03.2013.
        \end{innerlist}
\item[] \positionlocationdate{Erasmus}{\'Ecole normale sup\'erieure de Lyon, France}{2007}
\item[] \positionlocationdate{Diplom, Mathematics}{Technische Universit\"at M\"unchen, Germany}{2004 - 2009}
            \begin{innerlist}
                \item Passed with high distinction (1.0).
                \item Thesis title: \textit{Fast solvers for the Navier-Stokes equations on high Reynold numbers}.
                \item Department of Mathematics.
                \item The German Diplom is equivalent to a Master degree.
                \item Supervisor: M. Ulbrich.
            \end{innerlist}
\end{outerlist}

%%%%%%%%%%%%%%%%%%%%%%%%%% Awards %%%%%%%%%%%%%%%%%%%%%%%%%%%%%%%%
\sectionheader{Awards and Prizes}
\begin{itemize}
\item {\textbf{Wilkinson Prize for Numerical Software}}, 2015\\
    The Wilkinson prize for Numerical Software is a prestigous prize in scientific computing, awarded every four years to the authors of an outstanding piece of numerical software.
    In 2015, the prize was awarded to Funke and his co-authors P.E. Farrell, D.A. Ham and M.E. Rognes for the high-level AD tool dolfin-adjoint.
\item Best Poster Award, CSE 2015
\item Imperial College Startup Venture Catalyst Award, 2013
\item {Imperial College Excellence Award}, 2010%\\
%This award was received as part of the Applied Modelling And Computation Group, Imperial College London for its high academic achievements and significant future potential.
\item {Grantham Institute for Climate Change and Fujitsu CASE Studentship}, 2009
%    This PhD studentship was received to develop novel numerical methods on renewable energy using high-performance computing.
\item {Google Interactivism Award}, 2009
\item Hurwitz-Association Award for an excellence diploma thesis, 2009
\end{itemize}

%%%%%%%%%%%%%%%%%%%%%%%%%% Awards %%%%%%%%%%%%%%%%%%%%%%%%%%%%%%%%
\sectionheader{Funding}
\begin{itemize}
\item PI on NOK 7M FRIPRO Young Research Talent ``Simulation-based optimisation with dynamic domains'', (\#251237/F20), 2016
\item EU Horizon 2020 SME instrument, IMERSO AS, 2015
\end{itemize}

%%%%%%%%%%%%%%%%%%%%%%%%%% Mobility %%%%%%%%%%%%%%%%%%%%%%%%%%%%%%%%
\sectionheader{Mobility}
\begin{outerlist}
\item[] {Studied and worked in research institutions in \textbf{Germany, France, UK and Norway} (> 6 months).}
\item[] {Shorter research visits:
    \begin{innerlist}
        \item Heidelberg Laureate Forum, Germany (2016)
        \item Texas A\&M University, USA (2015)
        \item Humboldt University of Berlin, Germany (2013)
        \item Simula Research Laboratory, Norway (2012)
        \item Isaac Newton Institute, Cambridge, UK (2012)
        \item Institut of Atmospheric Physics, Beijing, China (2011).
    \end{innerlist}
        }
\end{outerlist}

%%%%%%%%%%%%%%%%%%%%%%%%%% Supervision %%%%%%%%%%%%%%%%%%%%%%%%%%%%%%%%
\sectionheader{Supervision}
\begin{outerlist}
\item[] \supervision{PostDocs:}{August Johannsson (Simula)}{}
\item[] \supervision{PhD students:}{J\/orgen Dokken (Simula), David Culley (Imperial College, co-superviser)}{}
\end{outerlist}

\sectionheader{Teaching activities}
\begin{outerlist}
    \item[] \supervision{Lecturer,}{INF3331/INF4331 Higher-level programming, University of Oslo, Norway}{2015 - today}
    \item[] \supervision{Python Workshop}{NTNU Trondheim (2017, 2 days)}{2017 - today}
    \item[] \supervision{FEniCS/dolfin-adjoint Workshop}{UNISA Johannesburg (2016, 1 day), Technical University of Munich (2016, 5 days), SUURPH workshop, Simula (2016, 1 day), Simula (2016, 2 days), Simula (2014, 1 day), Zhejiang University (2014, 5 days)}{2014 - today}
%    \item[] \supervision{\hspace{1.2cm}Invited lecture,}{Introduction to PDE-constrained optimisation, Zhejiang University, China (1 week)}{2014}
    \item[] \supervision{Teaching assistant}{for courses on computational science, programming and mathematics. Department of Earth Science and Engineering, Imperial College London, UK}{2010 - 2012}
\end{outerlist}

\sectionheader{Commissions of Trust}
\begin{outerlist}
\item[] \supervision{Reviewer}{for SIAM Journal on Scientific Computing (SIAM),
    Computer Physics Communications (Elsevier), Energies (MDPI) and European Wave
and Tidal Energy Conference Series and Geoscientific Model Development (GMD), Applied Energy}{2013 - today}
\item[] \supervision{\textbf{Examiner}}{for 2 Master projects}{2015-today}
\item [] \supervision{\textbf{Organiser}}{of a Workshop on Advanced Techniques in Biomedical Computing, \textit{Center for Biomedical Computing}}{2015}
\item [] \supervision{\textbf{Organiser}}{of a Mini-symposium on Efficient Solvers for PDE-constrained Optimization, \textit{SIAM CSE15}}{2015}
\end{outerlist}

\sectionheader{Research Interest/Research Profile }
%Provide an ‘abstract’, no more than a short paragraph, which outlines your current and
%prospective lines of research.

Funke's research interests center around optimisation problems governed by
partial differential equations, with a focus on their numerical solution. In
particular, he is interested in the automated derivation of adjoint and tangent
linear models and their use in optimal control, data inversion and design optimisation.
He is a founder of the dolfin-adjoint project, a software which automatically
derives adjoint models from complex computer models solving partial
differential equations based on a high-level symbolic problem specification
language that mimics mathematical notation.

Funke applies these advances to applications in a wide variety of different
domains including renewable energy and bioengineering. For the renewable energy
sector, he developed OpenTidalFarm, an open-source software for optimising of
tidal turbine farms, such as the optimal position of turbines within the farm.
In bioengineering, he developes data assimilation techniques for blood flow
simulations, with the aim to tune high-fidelity numerical blood flow models to
match with MRI measurements.

%%%%%%%%%%%%%%%%%%%%%%%%%% Publications %%%%%%%%%%%%%%%%%%%%%%%%%%%%%%%%
\newpage
\sectionheader{Journal publications}
\begin{enumerate}
    \item \textit{SW Funke, SC Kramer, MD Piggott.} Design optimisation and resource assessment for tidal-stream renewable energy farms using a new continuous turbine approach, \textit{Renewable Energy}, \href{https://dx.doi.org/10.1016/j.renene.2016.07.039}{doi:10.1016/j.renene.2016.07.039}, 2016.
    \item \textit{DM Culley, SW Funke, SC Kramer, MD Piggott.} Integration of cost modelling within the micro-siting design optimisation of tidal turbine arrays, \textit{Renewable Energy}, \href{https://doi.org/10.1016/j.renene.2015.06.013}{doi:10.1016/j.renene.2015.06.013}, 2016.
    \item \textit{S Rao , H Xue, M Bao, SW Funke.} Determining tidal turbine farm efficiency in the Western Passage using the disc actuator theory, \textit{Ocean Dynamics}, \href{https://doi.org/10.1007/s10236-015-0906-y}{doi:10.1007/s10236-015-0906-y}, 2015
\item \textit{PE Farrell, A Birkisson, SW Funke.} Deflation techniques for finding distinct solutions of nonlinear partial differential equations, \textit{SIAM Journal on Scientific Computing}, \href{https://doi.org/10.1137/140984798}{doi:10.1137/140984798}, 2015.
\item \textit{R Venell, SW Funke, S Draper, C Stevens.} Designing Large Arrays of Tidal Turbines: a synthesis and review, \textit{Renewable \& Sustainable Energy Reviews}, \href{https://doi.org/10.1016/j.rser.2014.08.022}{doi:10.1016/j.rser.2014.08.022}, 2015.
\item \textit{SW Funke, PE Farrell, MD Piggott.} Tidal turbine array optimisation using the adjoint approach, \textit{Renewable Energy}, \href{https://doi.org/10.1016/j.renene.2013.09.031}{doi:10.1016/j.renene.2013.09.031}, 2014.
\item \textit{PE Farrell, CJ Cotter, SW Funke.} A framework for the automation of generalised stability theory. \textit{SIAM Journal on     Scientific Computing}, \href{https://doi.org/10.1016/10.1137/12090074}{doi:10.1137/12090074}, 2014.
%\item \textit{PE Farrell, SW Funke}. Exploiting high-level structure in algorithmic differentiation, submitted, 2013.
\item \textit{PE Farrell, DA Ham, SW Funke, ME Rognes.} Automated derivation of the adjoint of high-level transient finite element programs, \textit{SIAM Journal on Scientific Computing}, \href{https://doi.org/10.1016/10.1137/120873558}{doi:10.1137/120873558}, 2013.
\item \textit{SW Funke, CC Pain, SC Kramer, MD Piggott.} A wetting and drying algorithm with a combined pressure/free-surface formulation for non-hydrostatic models, \textit{Advances in Water Resources}, \href{https://doi.org/10.1016/j.advwatres.2011.08.007}{doi:10.1016/j.advwatres.2011.08.007}, 2011.
\end{enumerate}

\sectionheader{Conference publications}
\begin{enumerate}
    \item \textit{CT Jacobs, SC Kramer, MD Piggott, SW Funke.} On the validity of tidal turbine array configurations obtained from steady-state adjoint optimisation, \textit{ECCOMAS Congress 2016}, \href{https://github.com/funsim/cv/raw/master/publications/jacobs_eccomas_2016.pdf}{PDF}, 2016.
    \item \textit{DM Culley, SW Funke, SC Kramer, MD Piggott.} Tidal stream resource assessment through optimisation of array design with quantification of uncertainty, \textit{EWTEC 2015 proceedings}, \href{https://github.com/funsim/cv/raw/master/publications/culley_ewtec_2015.pdf}{PDF}, 2015.
    \item \textit{T Roc, SW Funke, KM Thyng.} Standard methodology for tidal array project optimisation: An idealized study of the Minas Passage, \textit{EWTEC 2015 proceedings}, \href{https://github.com/funsim/cv/raw/master/publications/roc_ewtec_2015.pdf}{PDF}, 2015.
    \item \textit{SC Kramer, SW Funke, MD Piggott.} A continuous approach for the optimisation of tidal turbine farms, \textit{EWTEC 2015 proceedings}, \href{https://github.com/funsim/cv/raw/master/publications/kramer_ewtec_2015.pdf}{PDF}, 2015.
    \item \textit{DM Culley, SW Funke, SC Kramer, MD Piggott.} A hierarchy of approaches for the optimal design of tidal turbine arrays, \textit{Proceedings of the 5th International Conference on Ocean Energy}, \href{https://github.com/funsim/cv/raw/master/publications/culley_ewtec_2015.pdf}{PDF}, 2014.
\end{enumerate}

\sectionheader{In review}
\begin{enumerate}
    \item \textit{MM Noack, SW Funke} Hybrid Genetic Deflated Newton Method for Global Optimisation, \textit{Computational and Applied Mathematics}, \href{}{}, 2017.
    \item \textit{SW Funke, PE Farrell, MD Piggott.} Reconstructing wave profiles from inundation data, \textit{Computer Methods in Applied Mechanics and Engineering}, \href{}{}, 2017.
    \item \textit{T Schwedes, DA Ham, SW Funke,  MD Piggott} Mesh dependence in PDE-constrained optimisation, \textit{Springer Research Brief}, \href{}{}, 2016.
    \item \textit{SW Funke, M Nordaas, \O~Evju, MS Aln{\ae}s, K-A Mardal.} Variational data assimilation for transient blood flow simulations, \textit{SISC}, 2016.
    \item \textit{SD Parkinson, SW Funke, J Hill, MD Piggott, PA Allison.} Application of the adjoint approach to optimise the initial conditions of a turbidity current (AdjointTurbidity 1.0), \textit{Geoscientific Model Development (GMD)}, 2016.

    \item \textit{DM Culley, SW Funke, SC Kramer, MD Piggott.} A surrogate-model assisted approach for optimising the size of tidal turbine arrays, \textit{International Journal of Marine Energy}, 2016.
\item \textit{Roan du Feu, SW Funke, SC Kramer, DM Culley, J Hill, BS Halpern, MD Piggott.} The trade off between tidal-turbine array yield and environmental impact: a multi-objective optimisation problem, \textit{Renewable Energy}, submitted, 2016.
%\item \textit{PE Farrell, SW Funke.} Exploiting high-level structure in algorithmic differentiation, in review, 2014.
%\item \textit{GL Barnett, SW Funke, MD Piggott.} Hybrid global-local optimisation algorithms for the layout design of tidal turbine arrays, \textit{Renewable Energy}, in review, 2014.
\end{enumerate}

%%%%%%%%%%%%%%%%%%%%%%%%%% Dissemination %%%%%%%%%%%%%%%%%%%%%%%%%%%%%%%%
\sectionheader{Dissemination}

Simon Funke attented various national and international conferences. The following list
presents a selection of presentations:
\begin{outerlist}
\item[] \textbf{Invited talk} Introduction to FEniCS and dolfin-adjoint, \textit{Symposium on the Application of Finite Elements in Physics and Engineering}, Johannesburg, South Africa, 2016
\item[] Designing Tidal Turbine Arrays With PDE-constrained Optimisation, \textit{ESCO2016}, Plze\v{n}, Czech Republic, 2016
\item[] \textbf{Best poster award}, dolfin-adjoint, automated adjoint models for FEniCS, \textit{SIAM Conference on Computational Science}, Salt Lake City, USA, 2015
\item[] Tidal Farm Layout Optimisation and Resource Assessment based on PDE-constrained optimisation, \textit{International Conference on Ocean Energy}, Halifax, Canada, 2014
\item[] \textbf{Invited talk}, Introduction to FEniCS and automated adjoints, \textit{Norwegian Meteorological Institute}, Oslo, Norway, 2014
%\item[] Automated adjoint models with FEniCS with applications in cardiac electrophysiology, \textit{Simula Research Laboratory}, Oslo, Norway, 2014
\item[] PDE-constrained optimisation in Hilbert spaces, \textit{FEniCS'14}, Paris, France, 2014
%\item[] Automated adjoints of finite element discretizations, \textit{SIAM Conference on Computational Science \& Engineering}, Boston, USA, 2013
%\item[] PDE-constrained optimisation using automated adjoints of finite element models, Imperial College London, London, UK, 2012
%\item[] libadjoint: a new abstraction for developing adjoint models, \textit{FEniCS'12}, Simula Research Laboratory, Oslo, Norway, 2012
%\item[] Tidal turbine optimisation using the adjoint approach, \textit{Ocean Day}, Imperial College London, London, UK, 2012
%\item[] \textbf{Invited talk}, A new way to develop discrete adjoints, Simula Research Laboratory, Oslo, Norway, 2012
%\item[] Towards a Fluidity-ICOM adjoint, \textit{International Workshop on Multiscale (Un-)structured Mesh Numerical Modelling}, Alfred Wegener Institute for Polar and Marine Research, Bremerhaven, Germany, 2011
\item[] \textbf{Invited talk} An introduction to libadjoint, Institut of Atmospheric Physics, Beijing, China, July 2011
%\item[] A wetting and drying algorithm for non-hydrostatic models with combined pressure/free-surface, \textit{European Geosciences Union General Assembly 2011}, Vienna, Austria, 2011.
%\item[] A new wetting and drying algorithm using a combined pressure/free-surface finite element method, \textit{International Workshop on Multiscale (Un)-structured Mesh Numerical Ocean Modeling}, Massachusetts Institute of Technology, Boston, USA, 2010.
\end{outerlist}

\sectionheader{Other prices}
\begin{outerlist}
\item[] Winner of BigDataHackthon for developing a machine learning algorithm that predicts the influence of people, May 2013.
\item[] Winner of the Rewired State Hack, for novel developments for a charity that aims to improve surgical care in Africa, December 2012.
\item[] Invitation to OpenITP \& Rio de Janeiro RightsCon with the goal to develop realtime censorship detection, May 2012.
\item[] Google Interactivism Award for developing a NHS health care web application that aims to improve waiting times in hospitals, 2012.
\item[] Winner of WaterHackathon London for developing a reporting platform for managing sanitation complaints in Tanzania. The project received media attention and was featured on the BBC, Africa, October 2011.
%\item[] Developed a SMS notification system for \url{LifeBox}{http://www.lifebox.org}
\end{outerlist}

\end{document}

%%%%%%%%%%%%%%%%%%%%%%%%%% End CV Document %%%%%%%%%%%%%%%%%%%%%%%%%%%%%
